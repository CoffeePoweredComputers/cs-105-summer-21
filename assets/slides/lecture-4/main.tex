\documentclass{beamer}

\usepackage{graphicx}
\usepackage{textpos}
\usepackage{listings}

\usetheme{Madrid}
\useoutertheme{miniframes} % Alternatively: miniframes, infolines, split

% Setup the university's color pallette
\definecolor{UIUCorange}{RGB}{19, 41, 75} % UBC Blue (primary)
\definecolor{UIUCblue}{RGB}{232, 74, 39} % UBC Grey (secondary)

\definecolor{codegreen}{rgb}{0,0.6,0}
\definecolor{codegray}{rgb}{0.5,0.5,0.5}
\definecolor{codepurple}{rgb}{0.58,0,0.82}
\definecolor{backcolour}{rgb}{0.95,0.95,0.92}

\lstdefinestyle{python}{
  backgroundcolor=\color{backcolour},   
  commentstyle=\color{codegreen},
  keywordstyle=\color{magenta},
  numberstyle=\tiny\color{codegray},
  stringstyle=\color{codepurple},
  basicstyle=\ttfamily\footnotesize,
  breakatwhitespace=false,         
  breaklines=true,                 
  captionpos=b,                    
  keepspaces=true,                 
  numbers=left,                    
  numbersep=5pt,                  
  showspaces=false,                
  showstringspaces=false,
  showtabs=false,                  
  tabsize=2
}

\lstset{style=python}

\setbeamercolor{palette primary}{bg=UIUCorange,fg=white}
\setbeamercolor{palette secondary}{bg=UIUCblue,fg=white}
\setbeamercolor{palette tertiary}{bg=UIUCblue,fg=white}
\setbeamercolor{palette quaternary}{bg=UIUCblue,fg=white}
\setbeamercolor{structure}{fg=UIUCorange} % itemize, enumerate, etc
\setbeamercolor{section in toc}{fg=UIUCblue} % TOC sections

\setbeamercolor{subsection in head/foot}{bg=UIUCorange,fg=UIUCblue}
\setbeamercolor{subsection in head/foot}{bg=UIUCorange,fg=UIUCblue}

\usepackage[utf8]{inputenc}
\usepackage{graphicx}

%\usepackage[shortlabels]{enumitem}

%Information to be included in the title page:
\title{\textbf{Topic 3: Lists, Strings, Sets and Dictionaries}}
\author{\textbf{David H Smith IV}}
\institute[\textbf{UIUC}]{\textbf{University of Illinois Urbana-Champaign}}
\date{\textbf{Wed, June 23 2021}}

\setbeamertemplate{title page}[default][colsep=-4bp,rounded=true]
\addtobeamertemplate{title page}{\vspace{3\baselineskip}}{}
\addtobeamertemplate{title page}{
  \begin{textblock*}{\paperwidth}(-1.0em, -1.2em)
    \includegraphics[width=\paperwidth, height=\paperheight]{imgs/uiuc.png}
  \end{textblock*} 
}{}

\begin{document}

\frame{\titlepage}

\section{Announcements}

%
% Slide 1
%
\begin{frame}
  \frametitle{Announcements}
  \begin{enumerate}
    \item Quiz 1 is this Thursday:
      \begin{enumerate}
        \item https://ae3.engineering.illinois.edu/computer-based-testing-facility/information-for-students/exam-preparation/
        \item https://ae3.engineering.illinois.edu/computer-based-testing-facility/information-for-students/exam-preparation/
          https://ae3.engineering.illinois.edu/computer-based-testing-facility/information-for-students/exam-procedure/ 
        \item Links are in chat, on the course website, and on the uploaded slides for future reference
        \item \textbf{Make sure you have Zoom installed on your phone, tablet, or additional computer and are able to login with SSO}.
      \end{enumerate}
  \end{enumerate}
\end{frame}

\section{Strings}
%
% Slide 2
%
\begin{frame}[fragile]
  \frametitle{Poll Question: String Len}
  \begin{lstlisting}[language=Python] 
  my_str = "le chat"
  my_str_length = len(my_str)
  \end{lstlisting}
  \begin{enumerate}[A]
    \item 6
    \item 8
    \item 7 %correct
    \item 5
    \item 3
    \item 4
  \end{enumerate}
\end{frame}

%
% Slide 3
%
\begin{frame}[fragile]
  \frametitle{Poll Question: String Concatenation}
  \begin{enumerate}[A] 
    \item TypeError
    \item \lstinline{"New""York"} 
    \item \lstinline{"New+York"} 
    \item \lstinline{"New York"} 
    \item \lstinline{"NewYork"} 
  \end{enumerate}
\end{frame}

%
% Slide 4
%
\begin{frame}[fragile]
  \frametitle{Poll Question: String Concatenation}
  \begin{enumerate}[A] 
    \item \lstinline{"NewYork"}  %correct
    \item \lstinline{"New York"} 
    \item \lstinline{"New+York"} 
    \item TypeError
    \item \lstinline{"New""York"} 
  \end{enumerate}
\end{frame}

%
% Slide 6
%
\begin{frame}[fragile]
  \frametitle{Poll Question: Negative Indexing}
  What is the result of the following segment of code?
  \begin{lstlisting}[language=Python] 
  x = "ABCDE"
  print(x[-3])
  \end{lstlisting}
  \vfill
  \begin{enumerate}[A] 
    \item SyntaxError
    \item IndexError
    \item \lstinline{'A'}
    \item \lstinline{'B'}
    \item \lstinline{'C'} %correct
    \item \lstinline{'D'}
  \end{enumerate}
\end{frame}

%
% Slide 6
%
\begin{frame}[fragile]
  \frametitle{Poll Question: Changing Strings}
  What is the value of \lstinline|str1| after the following segment of code executes?
  \begin{lstlisting}[language=Python] 
  str1 = "urbana"
  str1[0] = "e"
  \end{lstlisting}
  \vfill
  \begin{enumerate}[A] 
    \item SyntaxError
    \item IndexError
    \item TypeError %correct
    \item \lstinline{'erbana'}
    \item \lstinline{'eurbana'}
  \end{enumerate}
\end{frame}


\section{String Formatting}

\begin{frame}[fragile]
  \frametitle{List Functions}
  What is the value of x after the following code executes:
  \begin{lstlisting}[language=Python] 
  x = "{1} {0} {1}".format("S", "O", "Y")
  \end{lstlisting}
  \begin{enumerate}[A] 
    \item \lstinline|"S O Y"|
    \item \lstinline|"SOY"|
    \item \lstinline|"O S O"| %correct
    \item \lstinline|"Y O Y"|
    \item \lstinline|"OSO"|
  \end{enumerate}
\end{frame}

\section{Lists}
%
% Slide 7
%
\begin{frame}[fragile]
  \frametitle{Poll Question: Indexing with Lists}
  \begin{lstlisting}
  \end{lstlisting}
  \begin{enumerate}[A] 
    \item 
  \end{enumerate}
\end{frame}



%
% Slide 8
%
\begin{frame}[fragile]
  \frametitle{Poll Question: }
  What does \lstinline|list1| contain after the code above executes?
  \begin{lstlisting}[language=Python]
  list1 = [1, 2]
  list1.append([3, 4])
  \end{lstlisting}
  \vfill
  \begin{enumerate}[A] 
    \item \lstinline|[1, 2]|
    \item \lstinline|[1, 2, 3, 4]|
    \item \lstinline|[1, 2, [3, 4]]| %correct
    \item \lstinline|[[1, 2], [3, 4]]|
  \end{enumerate}
  \
\end{frame}


%
% Slide 9
%
\begin{frame}[fragile]
  \frametitle{Poll Question: Changing Lists}
  After the following code executes what gets outputted to the screen?
  \begin{lstlisting}[language=Python]
  list1 = ["a", 5, "c", 234, ["hello", "world"]]
  list[4] = "Hello, World!"
  print(list1)
  \end{lstlisting}
  \vfill
  \begin{enumerate}[A] 
    \item \lstinline|['a', 5, 'c', 234, "Hello, World!"]| %correct
    \item TypeError
    \item SyntaxError
    \item \lstinline|['a', 5, 'c', 234, ["hello", "world", "Hello, World!"]]|
    \item \lstinline|['a', 5, 'c', 234, ["Hello, World!"]]|
  \end{enumerate}
\end{frame}

\section{Sets}

%
% Slide 10
%
\begin{frame}[fragile]
  \frametitle{Poll Question: Making Sets}
  Which of the follow are valid ways to construct a set?
  \begin{enumerate}[A] 
    \item \lstinline|x = set(1, 2, 3)|
    \item \lstinline|x = set([1, 2, 3])|
    \item \lstinline|x = [1, 2, 3, 4]|
    \item \lstinline|x = {1, 2, 3, 4}|
    \item \lstinline|x = {[1, 2, 3, 4]}|
  \end{enumerate}
  \pause
  \vfill
  %\hline
  \vfill
  \begin{enumerate}
    \item \textbf{set()} \textrightarrow Creates either a new set by either accepting and converting another sequence type (e.g., list, tuple) or creates an empty list if it's not given anything.
    \item \textbf{set literal} \textrightarrow Written using the \lstinline|{}|.
  \end{enumerate}
\end{frame}


%
% Slide 11
%
\begin{frame}[fragile]
  \frametitle{Poll Question: Set Tracing}
  \begin{lstlisting}[language=Python]
  set1 = {""}
  set2.add(2)
  \end{lstlisting}
  \vfill
  \begin{enumerate}[A] 
    \item 
  \end{enumerate}
\end{frame}

\section{Dictionaries}

%
% Slide 12
%
\begin{frame}[fragile]
  \frametitle{Poll Question: }
  \begin{enumerate}[A] 
    \item 
  \end{enumerate}
\end{frame}


%
% Slide 13
%
\begin{frame}[fragile]
  \frametitle{Poll Question: }
  \begin{enumerate}[A] 
    \item 
  \end{enumerate}
\end{frame}

\section{String Formatting}

\begin{frame}[fragile]
  \frametitle{List Functions}
  \begin{enumerate}[A] 
    \item 
  \end{enumerate}
\end{frame}


\end{document}
