\documentclass{beamer}

\usepackage{graphicx}
\usepackage{textpos}
\usepackage{listings}
\usepackage{lstautogobble}

\usetheme{Madrid}
\useoutertheme{miniframes} % Alternatively: miniframes, infolines, split

% Setup the university's color pallette
\definecolor{UIUCorange}{RGB}{19, 41, 75} % UBC Blue (primary)
\definecolor{UIUCblue}{RGB}{232, 74, 39} % UBC Grey (secondary)

\definecolor{codegreen}{rgb}{0,0.6,0}
\definecolor{codegray}{rgb}{0.5,0.5,0.5}
\definecolor{codepurple}{rgb}{0.58,0,0.82}
\definecolor{backcolour}{rgb}{0.95,0.95,0.92}

\lstdefinestyle{python}{
  backgroundcolor=\color{backcolour},   
  commentstyle=\color{codegreen},
  keywordstyle=\color{magenta},
  numberstyle=\tiny\color{codegray},
  stringstyle=\color{codepurple},
  basicstyle=\ttfamily\footnotesize,
  breakatwhitespace=false,         
  belowskip=-0.5em,
  breaklines=true,                 
  captionpos=b,                    
  keepspaces=true,                 
  numbers=left,                    
  numbersep=5pt,                  
  showspaces=false,                
  showstringspaces=false,
  showtabs=false,                  
  tabsize=2
}

\lstset{style=python}

\AtBeginSection[]{
    \begin{frame}
        \vfill
        \centering
        \begin{beamercolorbox}[sep=8pt,center,shadow=true,rounded=true]{title}
            \usebeamerfont{title}\insertsectionhead\par%
        \end{beamercolorbox}
        \vfill
    \end{frame}
}
% Setup the university's color pallette
\definecolor{UIUCorange}{RGB}{19, 41, 75} % UBC Blue (primary)
\definecolor{UIUCblue}{RGB}{232, 74, 39} % UBC Grey (secondary)


\setbeamercolor{palette primary}{bg=UIUCorange,fg=white}
\setbeamercolor{palette secondary}{bg=UIUCblue,fg=white}
\setbeamercolor{palette tertiary}{bg=UIUCblue,fg=white}
\setbeamercolor{palette quaternary}{bg=UIUCblue,fg=white}
\setbeamercolor{structure}{fg=UIUCorange} % itemize, enumerate, etc
\setbeamercolor{section in toc}{fg=UIUCblue} % TOC sections

\setbeamercolor{subsection in head/foot}{bg=UIUCorange,fg=UIUCblue}
\setbeamercolor{subsection in head/foot}{bg=UIUCorange,fg=UIUCblue}

\usepackage[utf8]{inputenc}
\usepackage{graphicx}

%Information to be included in the title page:
\title{\textbf{Functions and Excel}}
\author{\textbf{David H Smith IV}}
\institute[\textbf{UIUC}]{\textbf{University of Illinois Urbana-Champaign}}
\date{\textbf{Mon, July 12 2021}}

\setbeamertemplate{title page}[default][colsep=-4bp,rounded=true]
\addtobeamertemplate{title page}{\vspace{3\baselineskip}}{}
\addtobeamertemplate{title page}{
  \begin{textblock*}{\paperwidth}(-1.0em, -1.2em)
    \includegraphics[width=\paperwidth, height=\paperheight]{imgs/uiuc.png}
  \end{textblock*} 
}{}

\begin{document}

\frame{\titlepage}

\section{Announcements}

%
% Slide 1
%
\begin{frame}
  \frametitle{Announcements}
  \begin{itemize}
    \item 
  \end{itemize}
\end{frame}

\section{More on Functions}

%
% Slide 2
%
\begin{frame}[fragile]
  \frametitle{Dynamic Typing}
  Python is a dynamically typed language
  \begin{itemize}
    \item The \lstinline|x| does several different things:
      \begin{enumerate}
        \item Integer addition
        \item Floating point addition
        \item String concatenation
      \end{enumerate}
    \item At \textbf{runtime} Python looks at the \lstinline|+| operator and determines the correct behaviours based on the types or it's operands.
    \item \textbf{Polymorphism} \textrightarrow A single piece of code can do several things depending on the type of data it's working with.
      \begin{enumerate}
        \item You can write less code.
        \item Can be harder to find bugs.
      \end{enumerate}

  \end{itemize}
\end{frame}

%
% Slide 2
%
\begin{frame}[fragile]
  \frametitle{Poll Question: Polymorphic Functions}
  \begin{lstlisting}[language=Python, autogobble]
  def print_all(collection):
    for item in collection:
      print(item)

  print_all([1, 2, 3, 4])
  print_all({ 'k': 'v', 'CS': '105' }) #A
  print_all(7)                         #B
  print_all('a string')                #C
  \end{lstlisting}
  \vfill
  \begin{enumerate}[A]
    \item A error
    \item B error
    \item C error
    \item A \& B error
    \item A \& C error
    \item B \& C error
  \end{enumerate}
\end{frame}


%
% Slide 2
%
\begin{frame}[fragile]
  \frametitle{Poll Question: Function Scoping}
  \begin{lstlisting}[language=Python, autogobble]
  my_var = 11
  def my_print(my_var):
    print(my_var)

  my_print(22)
  print(my_var)
  \end{lstlisting}
  \vfill
  \begin{enumerate}[A]
    \item 11\\11
    \item 11\\22
    \item 22\\11
    \item 22\\22
    \item NameError
  \end{enumerate}
\end{frame}


%
% Slide 2
%
\begin{frame}[fragile]
  \frametitle{Scoping}
  \begin{itemize}
    \item Every function is given a clean slate.
    \item Any variables written in a function are defined in the function's scope.
    \item The scope is destroyed when the function returns.
    \item If a name is read that doesn't exist in the function's scope, it tries the scope the function was defined in.
  \end{itemize}
\end{frame}


\section{Excel Polls}
%
% Slide 2
%
\begin{frame}[fragile]
  \frametitle{Poll Question: Excel Cell Range}
  How many cells in region \lstinline|A1:B10|
  \vfill
  \begin{enumerate}[A]
    \item 9
    \item 10
    \item 18
    \item 20
  \end{enumerate}
  \pause
  \textbf{Answer: } Colon-separated pair follows the form \lstinline|upper-left:lower:right|.
\end{frame}


%
% Slide 2
%
\begin{frame}[fragile]
  \frametitle{Poll Question: Excel}
  If cell A1 contains \lstinline|Champaign|\\
  \begin{enumerate}
    \item \lstinline|=A1="CHAMPAIGN"|
    \item \lstinline|=A1>"Boston"|
    \item \lstinline|=A1="*ham*"|
  \end{enumerate}
  \vfill
  \begin{enumerate}[A]
    \item 1
    \item 2
    \item 3
    \item Both 1 \& 2
    \item Both 1 \& 3
    \item Both 2 \& 3
  \end{enumerate}
\end{frame}

\section{Excel Review}
%
% Slide 2
%
\begin{frame}[fragile]
  \frametitle{Excel: SUM}
  \begin{lstlisting}[language=Python, autogobble]
  def SUM(selected_range):
    total = 0
    for cell in selected_range:
      total += cell
    return total
  \end{lstlisting}
\end{frame}


%
% Slide 2
%
\begin{frame}[fragile]
  \frametitle{Excel: COUNTIF}
  \begin{lstlisting}[language=Python, autogobble]
  def COUNTIF(selected_range, criteria):
    count = 0
    for cell in selected_range:
      if meets_criteria(cell, criteria):
        count += 1
    return count
  \end{lstlisting}
  \vfill
  Criteria should be:
  \begin{itemize}
    \item A single value: \lstinline|7| or \lstinline|"Illinois"|
    \item A comparison: \lstinline|">7"| or \lstinline|">" \& B2|
    \item A string with wildcards: \lstinline|"*BADM*"| or \lstinline|"CS 10?"|
  \end{itemize}
\end{frame}


%
%
%
\begin{frame}[fragile]
  \frametitle{Excel: COUNTIF cont'd}
  \begin{enumerate}[A]
    \item \lstinline|COUNTIFS(range1, criteria1, range2, criteria2)|
    \item Ranges 1 and 2 must be of equal \# of rows and columns.
    \item Counts each time the nth value of each range meets that range's criteria.
  \end{enumerate}
\end{frame}

%
%
%
\begin{frame}[fragile]
  \frametitle{Poll Question: COUNTIF}
  How many rows are selected as a result of the following query?
  \lstinline|=COUNTIFS(A1:A7, ">10", B1:B7, "empty")|
  \vfill
  \begin{minipage}{0.48\textwidth}
  \begin{enumerate}[A]
    \item \lstinline|1|
    \item \lstinline|2|
    \item \lstinline|3|
    \item \lstinline|4|
    \item \lstinline|5|
  \end{enumerate}
  \end{minipage}
  \hfill
  \begin{minipage}{0.48\textwidth}
    \includegraphics[width=0.75\textwidth]{./imgs/spreadsheet_countif.png}
  \end{minipage}
\end{frame}


%
% Slide 2
%
\begin{frame}[fragile]
  \frametitle{Excel: INDEX}
  \begin{enumerate}
    \item 1-dimensional lookup:
      \begin{itemize}
        \item \lstinline|=INDEX(cellrange, rowindex)|
        \item \lstinline|=INDEX(B3:B11, 4)| \textrightarrow C6
      \end{itemize}
    \item 2-dimensional lookup:
      \begin{itemize}
        \item \lstinline|=INDEX(cellrange, rowindex, colindex)|
        \item \lstinline|=INDEX(B3:D11, 4, 2)| \textrightarrow C6
      \end{itemize}
  \end{enumerate}
\end{frame}


%
% Slide 2
%
\begin{frame}[fragile]
  \frametitle{Poll Question: INDEX}
  Which cell does \lstinline|=INDEX(C10:F20, 3, 2)| read from?
  \vfill
  \begin{enumerate}[A]
    \item D12
    \item E11
    \item E12
    \item E13
    \item F12
  \end{enumerate}
\end{frame}


%
% Slide 2
%
\begin{frame}[fragile]
  \frametitle{Excel: MATCH}
  \lstinline|=MATCH(value, cellrange, matchtype)|
  \begin{itemize}
    \item value is the value we're looking for
    \item cell-range is a row or a column 
    \item match type is:
      \begin{itemize}
        \item 0 = exact match
        \item 1 = Largest less than or equal, values must be sorted in increasing order
        \item -1 = Smallest value greater than or equal, values must be sorted in descending order
      \end{itemize}
    \item returns 1-based index into the cell range for a cell containing the value
    \item return \#N/A if no match
  \end{itemize}
\end{frame}


%
% Slide 2
%
\begin{frame}[fragile]
  \frametitle{Poll Question: MATCH}
  What does this call to the MATCH function return: \lstinline|=MATCH("Craig", A2:A8, 0)|.\\
  \vfill
  \begin{minipage}{0.48\textwidth}
    \begin{enumerate}[A]
      \item 2
      \item 3
      \item 4
      \item A3
      \item A4
      \item \#NA
    \end{enumerate}
  \end{minipage}
  \hfill
  \begin{minipage}{0.48\textwidth}
    \includegraphics[width=0.75\textwidth]{./imgs/spreadsheet_match.png}
  \end{minipage}
\end{frame}


%
% Slide 2
%
\begin{frame}[fragile]
  \frametitle{Excel: VLOOKUP - Combining INDEX and MATCH}
  \begin{itemize}
    \item Handles a common case of INDEX + MATCH
    \item \lstinline|=VLOOKUP(value, table, col_index, [range_lookup])|
      \begin{itemize}
        \item \textbf{value} \textrightarrow  Value to look for in the first column of table.
        \item \textbf{table} \textrightarrow  Table from which to retrieve a value.
        \item \textbf{col\_index} \textrightarrow The column in the table from which to retrieve a value.
        \item \textbf{range\_lookup} \textrightarrow TRUE for approximate match and FALSE for exact match.
      \end{itemize}
      \item Value searched for must be to the left of value to return.
      \item HLOOKUP does for rows, what VLOOKUP does for cols.
  \end{itemize}
\end{frame}


%
% Slide 2
%
\begin{frame}[fragile]
  \frametitle{Docstrings!}
  \centering
  Off to Repl.it we go\ldots
\end{frame}

\end{document}
