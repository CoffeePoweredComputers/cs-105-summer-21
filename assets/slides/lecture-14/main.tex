\documentclass{beamer}

\usepackage{graphicx}
\usepackage{textpos}
\usepackage{listings}
\usepackage{lstautogobble}

\usetheme{Madrid}
\useoutertheme{miniframes} % Alternatively: miniframes, infolines, split

% Setup the university's color pallette
\definecolor{UIUCorange}{RGB}{19, 41, 75} % UBC Blue (primary)
\definecolor{UIUCblue}{RGB}{232, 74, 39} % UBC Grey (secondary)

\definecolor{codegreen}{rgb}{0,0.6,0}
\definecolor{codegray}{rgb}{0.5,0.5,0.5}
\definecolor{codepurple}{rgb}{0.58,0,0.82}
\definecolor{backcolour}{rgb}{0.95,0.95,0.92}

\lstdefinestyle{python}{
  backgroundcolor=\color{backcolour},   
  commentstyle=\color{codegreen},
  keywordstyle=\color{magenta},
  numberstyle=\tiny\color{codegray},
  stringstyle=\color{codepurple},
  basicstyle=\ttfamily\footnotesize,
  breakatwhitespace=false,         
  belowskip=-0.5em,
  breaklines=true,                 
  captionpos=b,                    
  keepspaces=true,                 
  numbers=left,                    
  numbersep=5pt,                  
  showspaces=false,                
  showstringspaces=false,
  showtabs=false,                  
  tabsize=2,
  firstnumber=1
}

\lstset{style=python}

\AtBeginSection[]{
    \begin{frame}
        \vfill
        \centering
        \begin{beamercolorbox}[sep=8pt,center,shadow=true,rounded=true]{title}
            \usebeamerfont{title}\insertsectionhead\par%
        \end{beamercolorbox}
        \vfill
    \end{frame}
}
% Setup the university's color pallette
\definecolor{UIUCorange}{RGB}{19, 41, 75} % UBC Blue (primary)
\definecolor{UIUCblue}{RGB}{232, 74, 39} % UBC Grey (secondary)


\setbeamercolor{palette primary}{bg=UIUCorange,fg=white}
\setbeamercolor{palette secondary}{bg=UIUCblue,fg=white}
\setbeamercolor{palette tertiary}{bg=UIUCblue,fg=white}
\setbeamercolor{palette quaternary}{bg=UIUCblue,fg=white}
\setbeamercolor{structure}{fg=UIUCorange} % itemize, enumerate, etc
\setbeamercolor{section in toc}{fg=UIUCblue} % TOC sections

\setbeamercolor{subsection in head/foot}{bg=UIUCorange,fg=UIUCblue}
\setbeamercolor{subsection in head/foot}{bg=UIUCorange,fg=UIUCblue}

\usepackage[utf8]{inputenc}
\usepackage{graphicx}

%Information to be included in the title page:
\title{\textbf{Review Lecture}}
\author{\textbf{David H Smith IV}}
\institute[\textbf{UIUC}]{\textbf{University of Illinois Urbana-Champaign}}
\date{\textbf{Mon, Aug 02 2021}}

\setbeamertemplate{title page}[default][colsep=-4bp,rounded=true]
\addtobeamertemplate{title page}{\vspace{3\baselineskip}}{}
\addtobeamertemplate{title page}{
  \begin{textblock*}{\paperwidth}(-1.0em, -1.2em)
    \includegraphics[width=\paperwidth, height=\paperheight]{imgs/uiuc.png}
  \end{textblock*} 
}{}

\begin{document}

\frame{\titlepage}

\section{Reminders}

%
% Slide 1
%
\begin{frame}
  \frametitle{Reminders}
  \begin{itemize}
    \item Practice final is up and attempting it is the final homework (250 pts).
    \item The final ``Lecture'' is going to be an office hours where you can swing by and attempt the practice final.
    \item The final is this Saturday at 1pm so please register for it on the scheduler.
    \item I'll be available all week for questions, even outside of the usual office hours, so feel free to ping me on Discord.
    \item Please review the gradebook to make sure everything looks correct.
    \item \textbf{Please be sure to fill out the ICES forms if you haven't already.}
  \end{itemize}
\end{frame}

\section{Patterns}

\begin{frame}
  \frametitle{Patterns of Interest}
  \begin{itemize}
    \item Counting pattern
    \item Sum/Total Pattern
    \item Computing a Sum/Total Over Specific Elements
    \item Finding single item in collection
    \item Finding best in collection
    \item Filtering a collection
    \item Histogram
  \end{itemize}
\end{frame}

%
% Slide 2
%
\begin{frame}[fragile]
  \frametitle{Counting Pattern}
  \begin{lstlisting}[language=Python, autogobble][language=Python, autogobble]
  def count(collection):
    counter = 0
    for item in collection:
      if <item meets condition>:
        counter += 1
    return counter
  \end{lstlisting}
\end{frame}

%
% Slide 2
%
\begin{frame}[fragile]
  \frametitle{Computing a Sum/Total}
  \begin{lstlisting}[language=Python, autogobble][language=Python, autogobble]
  def sum(collection):
    total = 0
    for item in collection:
      total += item
    return total
  \end{lstlisting}
\end{frame}


%
% Slide 2
%
\begin{frame}[fragile]
  \frametitle{Computing a Sum/Total Over Specific Elements}
  \begin{lstlisting}[language=Python, autogobble][language=Python, autogobble]
  def sum(collection):
    total = 0
    for item in collection:
      if <condition>:
        total += item
    return total
  \end{lstlisting}
\end{frame}

%
% Slide 2
%
\begin{frame}[fragile]
  \frametitle{Dictionaries: Computing a Histogram}
  Creating a count map of items in a collection is a common dictionary pattern:
  \begin{lstlisting}[language=Python, autogobble]
  def get_item_counts(some_list):
    counts = {}
    for item in some_list:
      if item not in counts:
        counts[item] = 1
      else:
        counts[item] += 1
  \end{lstlisting}
\end{frame}

%
% Slide 2
%
\begin{frame}[fragile]
  \frametitle{Finding (single thing) in a Collection}
  \begin{lstlisting}[language=Python, autogobble][language=Python, autogobble]
  def find_thing(collection):
    for thing in collection:
      if <thing meets condition>:
        return thing
  \end{lstlisting}
  \vfill
  \begin{lstlisting}[language=Python, autogobble][language=Python, autogobble]
  def find_thing(collection):
    found = None
    for thing in collection:
      if <thing meets condition>:
        found = thing
        break
    return found
  \end{lstlisting}
\end{frame}

%
% Slide 2
%
\begin{frame}[fragile]
  \frametitle{Finding best in collection}
  \begin{lstlisting}[language=Python, autogobble][language=Python, autogobble]
  def find_best(collection):
    currentbest = ??
    for thing in collection:
      if <thing is better than current best>:
        currentbest = thing
    return currentbest 
  \end{lstlisting}
  \vfill
  \begin{itemize}
    \item If we're searching over a list and we want to return the largest or smaller number: \lstinline|currentbest = stufflist[0]| \pause
    \item If we're searching over a list of strings and we want to return the longest string: \lstinline|currentbest = stufflist[0]| or \lstinline|currentbest = ""| \pause
    \item If you know the list contains only non-negative integers: \lstinline|currentbest = -1|
  \end{itemize}
\end{frame}

%
% Slide 2
%
\begin{frame}[fragile]
  \frametitle{Filtering a collection}
  \begin{lstlisting}[language=Python, autogobble][language=Python, autogobble]
  def filter(collection):
    new_list = []

    for thing in collection:
      if <thing meets criteria>:
        newlist.append(thing)

    return new_list
  \end{lstlisting}
\end{frame}

\section{General Review: Common Points of Error}

%
% Slide 2
%
\begin{frame}[fragile]
  \frametitle{Poll Questions: Functions}
  How many functions will the function call \lstinline|count_letters\("Hello, 105!"\)| iterate?
  \begin{lstlisting}[language=Python, autogobble][language=Python, autogobble]
  def count_letters(s):
    count = 0
    for c in s:
      count += 1
      return count 
  \end{lstlisting}
  \vfill
  \begin{enumerate}[A]
    \item 0
    \item 1
    \item 10
    \item 11
  \end{enumerate}
\end{frame}

%
% Slide 2
%
\begin{frame}[fragile]
  \frametitle{Poll Question: Find and Slicing}
  What does the following code produce?
  \begin{lstlisting}[language=Python, autogobble][language=Python, autogobble]
  def foo(x):
    f = x.find(',')
    s = x.find(',', s + 1)
    return x[f:(s + 1)]
  foo("This, small sentence, is a test.")
  \end{lstlisting}
  \vfill
  \begin{enumerate}[A]
    \item Error on Line 2 
    \item Error on line 3 
    \item Error on line 4
    \item `, small sentence,' 
  \end{enumerate}
\end{frame}

%
% Slide 2
%
\begin{frame}[fragile]
  \frametitle{Poll Question: Find and Slicing}
  Which of these lines of code need to be used to fix it such that it produces the output \lstinline|, small sentence,|?
  \begin{lstlisting}[language=Python, autogobble][language=Python, autogobble]
  def foo(x):
    f = x.find(',')
    s = x.find(',', s + 1)
    return x[f:(s + 1)]
  foo("This, small sentence, is a test.")
  \end{lstlisting}
  \vfill
  \begin{enumerate}[A]
    \item 3: \lstinline|s = x.find(',', s + 1)|
    \item 3: \lstinline|s = x.find(',', f)|
    \item 3: \lstinline|s = x.find(',', f + 1)|
    \item 3: \lstinline|s = x.find(',', f - 1)|
  \end{enumerate}
\end{frame}

%
% Slide 2
%
\begin{frame}[fragile]
  \frametitle{Poll Question: Functions}
  What does the following function produce if called with \lstinline|foo(["apples", "bananas"])|?
  \begin{lstlisting}[language=Python, autogobble][language=Python, autogobble]
  def foo(str_list):
    count = 0
    for s in str_list
      for c in s:
        if c == 'a':
          count += 1
  return count
  \end{lstlisting}
  \vfill
  \begin{enumerate}[A]
    \item None
    \item Error
    \item 2
    \item 4
  \end{enumerate}
\end{frame}

%
% Slide 2
%
\begin{frame}[fragile]
  \frametitle{Filtering a collection}
  What is the value of new\_list after running the following code?
  \begin{lstlisting}[language=Python, autogobble][language=Python, autogobble]
  x = ["This", "Is", "A", "Test"]
  new_list = []
  for i in x:
    if i < 2:
      new_list.append(i)
  \end{lstlisting}
  \vfill
  \begin{enumerate}[A]
    \item Error
    \item \lstinline|["Is", "A"]|
    \item \lstinline|["A"]|
    \item \lstinline|[]|
  \end{enumerate}
\end{frame}

\section{Files}

%
% Slide 2
%
\begin{frame}[fragile]
  \frametitle{Files}
  Which of the following reads all the contents of a file into a list of strings?
  \vfill
  \begin{enumerate}[A]
    \item readlines
    \item readall
    \item read
    \item readline
  \end{enumerate}
\end{frame}


%
% Slide 2
%
\begin{frame}[fragile]
  \frametitle{Poll Question: Read Characters}
  Given a variable named \lstinline|file_object| that contains a file object which of the following will read the next 15 character into a variable named \lstinline|title|.
  \vfill
  \begin{enumerate}[A]
    \item \lstinline| title = file_object.read(15)|
    \item \lstinline| title = file_object.read(14)|
    \item \lstinline| title = file_object.reads(15)|
    \item \lstinline| title = read(file_object, 15)|
  \end{enumerate}
\end{frame}

%
% Slide 2
%
\begin{frame}[fragile]
  \frametitle{Reading from Files}
  Method 1:
  \begin{lstlisting}[language=Python, autogobble]
  file_object = open('filename')
  lines = file_object.readlines()
  for line in lines:
    print(line)
  file_object.close()
  \end{lstlisting}
  \vfill
  Method 2:
  \begin{lstlisting}[language=Python, autogobble]
  with open('filename') as inf:
    lines = inf.readlines()
    for line in lines:
      print(line)
    #automatic file close
  \end{lstlisting}
\end{frame}

%
% Slide 2
%
\begin{frame}[fragile]
  \frametitle{Writing to Files}
  Method 1:
  \begin{lstlisting}[language=Python, autogobble]
  file_object = open('filename', 'w')
  file_object.write('thing to write')
  file_objet.close()  #automatic at program end
  file_object.flush() #optional
  \end{lstlisting}
  \vfill
  Method 2:
  \begin{lstlisting}[language=Python, autogobble]
  with open('filename', 'w') as outf:
    outf.write('thing to write')
    #automatic file close
  \end{lstlisting}
\end{frame}

%
% Slide 2
%
\begin{frame}[fragile]
  \frametitle{Patterns and Files}
  \begin{minipage}{0.38\textwidth}
    Usual Sum/Total:
    \begin{lstlisting}[language=Python, autogobble]
    def foo(some_list):
      total = 0
      for item in some_list
        total += item
      return total
    \end{lstlisting}
  \end{minipage}
  \hfill
  \begin{minipage}{0.58\textwidth}
    Sum/Total Pattern w/ File:
    \begin{lstlisting}[language=Python, autogobble]
    def foo(filename):
      file_object = open(filename)
      lines = file_object.readlines()
      total = 0
      for line in lines:
        total += int(line)
      return total
    \end{lstlisting}
  \end{minipage}
\end{frame}

\section{More Review}

%
% Slide 2
%
\begin{frame}[fragile]
  \frametitle{Poll Question: Booleans}
  What is the correct boolean expression for if we want to determine if the list \lstinline|x| only if it contains either elements \lstinline|y| \lstinline|z|.
  \begin{enumerate}[A]
    \item \lstinline|y or z in x|
    \item \lstinline|x in y or x in z|
    \item \lstinline|y in x or z in x|
    \item \lstinline|y in x and z in x|
  \end{enumerate}
\end{frame}

%
% Slide 2
%
\begin{frame}[fragile]
  \frametitle{Filtering a collection}
  Which function call will produce the output \lstinline|`hj'|.
  \begin{lstlisting}[language=Python, autogobble]
  def foo(x, y, z):
      return x[y: z]
  \end{lstlisting}
  \vfill
  \begin{enumerate}[A]
    \item \lstinline|foo(`asdfghjkl', -4, -2)|
    \item \lstinline|foo(`asdfghjkl', -3, -2)|
    \item \lstinline|foo(`asdfghjkl', -3, -1)|
    \item \lstinline|foo(`asdfghjkl', -5, -2)|
  \end{enumerate}
\end{frame}

\end{document}
