\documentclass{beamer}

\usepackage{graphicx}
\usepackage{textpos}
\usepackage{listings}
\usepackage{lstautogobble}

\usetheme{Madrid}
\useoutertheme{miniframes} % Alternatively: miniframes, infolines, split

% Setup the university's color pallette
\definecolor{UIUCorange}{RGB}{19, 41, 75} % UBC Blue (primary)
\definecolor{UIUCblue}{RGB}{232, 74, 39} % UBC Grey (secondary)

\definecolor{codegreen}{rgb}{0,0.6,0}
\definecolor{codegray}{rgb}{0.5,0.5,0.5}
\definecolor{codepurple}{rgb}{0.58,0,0.82}
\definecolor{backcolour}{rgb}{0.95,0.95,0.92}

\lstdefinestyle{python}{
  backgroundcolor=\color{backcolour},   
  commentstyle=\color{codegreen},
  keywordstyle=\color{magenta},
  numberstyle=\tiny\color{codegray},
  stringstyle=\color{codepurple},
  basicstyle=\ttfamily\footnotesize,
  breakatwhitespace=false,         
  belowskip=-0.5em,
  breaklines=true,                 
  captionpos=b,                    
  keepspaces=true,                 
  numbers=left,                    
  numbersep=5pt,                  
  showspaces=false,                
  showstringspaces=false,
  showtabs=false,                  
  tabsize=2
}

\lstset{style=python}

\AtBeginSection[]{
    \begin{frame}
        \vfill
        \centering
        \begin{beamercolorbox}[sep=8pt,center,shadow=true,rounded=true]{title}
            \usebeamerfont{title}\insertsectionhead\par%
        \end{beamercolorbox}
        \vfill
    \end{frame}
}
% Setup the university's color pallette
\definecolor{UIUCorange}{RGB}{19, 41, 75} % UBC Blue (primary)
\definecolor{UIUCblue}{RGB}{232, 74, 39} % UBC Grey (secondary)


\setbeamercolor{palette primary}{bg=UIUCorange,fg=white}
\setbeamercolor{palette secondary}{bg=UIUCblue,fg=white}
\setbeamercolor{palette tertiary}{bg=UIUCblue,fg=white}
\setbeamercolor{palette quaternary}{bg=UIUCblue,fg=white}
\setbeamercolor{structure}{fg=UIUCorange} % itemize, enumerate, etc
\setbeamercolor{section in toc}{fg=UIUCblue} % TOC sections

\setbeamercolor{subsection in head/foot}{bg=UIUCorange,fg=UIUCblue}
\setbeamercolor{subsection in head/foot}{bg=UIUCorange,fg=UIUCblue}

\usepackage[utf8]{inputenc}
\usepackage{graphicx}

%Information to be included in the title page:
\title{\textbf{HTML}}
\author{\textbf{David H Smith IV}}
\institute[\textbf{UIUC}]{\textbf{University of Illinois Urbana-Champaign}}
\date{\textbf{Mon, July 21 2021}}

\setbeamertemplate{title page}[default][colsep=-4bp,rounded=true]
\addtobeamertemplate{title page}{\vspace{3\baselineskip}}{}
\addtobeamertemplate{title page}{
  \begin{textblock*}{\paperwidth}(-1.0em, -1.2em)
    \includegraphics[width=\paperwidth, height=\paperheight]{imgs/uiuc.png}
  \end{textblock*} 
}{}

\begin{document}

\frame{\titlepage}

\section{Reminders}

%
% Slide 1
%
\begin{frame}
  \frametitle{Reminders}
  \begin{itemize}
    \item This week is the last quiz. Don't forget to signup.
    \item Homework 7 is due Friday 
    \item The grace period for homework 6 ends Friday
    \item Practice quiz is up, be sure to attempt it before the actual quiz
  \end{itemize}
\end{frame}

\section{Lists}

%
% Slide 2
%
\begin{frame}[fragile]
  \frametitle{HTML5 = HTML, CSS, JS}
  \pause
  \begin{itemize}
    \item \textbf{Separation of Concerns: }
    \begin{itemize}
      \item HTML = Content
      \item CSS = styling
      \item JS = interactivity
    \end{itemize}
    \pause
    \item \textbf{HTML Documents are Heirarchical: }
    \begin{itemize}
      \item Elements have begin/end tags
      \item Elements can be nested in other elements
    \end{itemize}
    \pause
    \item \textbf{CSS consists of:}
    \begin{itemize}
      \item \{attribute : value\} pairs
      \item Like a Python dictionary
    \end{itemize}
    \pause
    \item \textbf{JS}
    \begin{itemize}
      \item variables, expressions, function, conditionals, loops, etc\ldots
    \end{itemize}
  \end{itemize}
\end{frame}

%
% Slide 2
%
\begin{frame}[fragile]
  \frametitle{Poll Question: HTML}
  Which statement is false?
  \begin{enumerate}[A]
    \item There are multiple versions of HTML and browsers support more than one of these versions simultaneously.
    \item HTML is a programming language like Python; it has variables, expressions, conditionals, loops and functions.
    \item It is hard to keep your web page design secret because all of the source (HTML, CSS, JS) is sent to the user's browser.
    \item None of the above
  \end{enumerate}
\end{frame}

%
% Slide
%
\begin{frame}[fragile]
  \frametitle{Poll Question: HTML}
  Which statement is false?
  \begin{itemize}
    \item Newlines and indentation in HTML are ignored.
    \item There is more than one type of list in HTML.
    \item Image tags don't require closing tags.
    \item ``alt attributes'' for images help make web pages accessible.
    \item ``favicon'' is a small picture to represent your web page, often shown in the browser tab.
    \item None of the above
  \end{itemize}
\end{frame}

%
% Slide 2
%
\begin{frame}[fragile]
  \frametitle{Poll Question: HTML}
  Which statement is false?
  \begin{enumerate}[A]
    \item HTML link have 2 main parts: the URL and the anchor text
    \item HTML links can go to the middle of a web page
    \item A relative link specifies the complete URL for a web page
    \item HTML needs ``escape sequences'' to specify certain characters like Python does
    \item \lstinline|&nbsp;| specifies a ```non-breaking space'' in HTML
    \item None of the above
  \end{enumerate}
\end{frame}

%
% Slide 2
%
\begin{frame}[fragile]
  \frametitle{Poll Question: HTML}
  Most of the web pages on the internet are\ldots
  \begin{enumerate}[A]
    \item Are written by humans from scratch
    \item Are generated by computer programs
  \end{enumerate}
\end{frame}

%
% Slide 2
%
\begin{frame}[fragile]
  \frametitle{Generating Web pages Dynamically}
  \begin{enumerate}[A]
    \item Amazon.com doesn't have people write web pages for each product. They're generated on the fly by a computer program according to a template.
    \item \textbf{Templates: } Basically like Python format strings, like \lstinline|"Product: {} Price: {}".format(productname, product[productname])|.
    \item Looping through collections
  \end{enumerate}
\end{frame}

%
% Slide 2
%
\begin{frame}[fragile]
  \frametitle{Reading data from the internet}
  \begin{enumerate}[A]
    \item From a Python program
    \item urllib module: Given a URL, returns the document at that URL
  \end{enumerate}
\end{frame}

\end{document}
