\documentclass{beamer}

\usepackage{graphicx}
\usepackage{textpos}
\usepackage{listings}
\usepackage{lstautogobble}

\usetheme{Madrid}
\useoutertheme{miniframes} % Alternatively: miniframes, infolines, split

% Setup the university's color pallette
\definecolor{UIUCorange}{RGB}{19, 41, 75} % UBC Blue (primary)
\definecolor{UIUCblue}{RGB}{232, 74, 39} % UBC Grey (secondary)

\definecolor{codegreen}{rgb}{0,0.6,0}
\definecolor{codegray}{rgb}{0.5,0.5,0.5}
\definecolor{codepurple}{rgb}{0.58,0,0.82}
\definecolor{backcolour}{rgb}{0.95,0.95,0.92}

\lstdefinestyle{python}{
  backgroundcolor=\color{backcolour},   
  commentstyle=\color{codegreen},
  keywordstyle=\color{magenta},
  numberstyle=\tiny\color{codegray},
  stringstyle=\color{codepurple},
  basicstyle=\ttfamily\footnotesize,
  breakatwhitespace=false,         
  belowskip=-0.5em,
  breaklines=true,                 
  captionpos=b,                    
  keepspaces=true,                 
  numbers=left,                    
  numbersep=5pt,                  
  showspaces=false,                
  showstringspaces=false,
  showtabs=false,                  
  tabsize=2
}

\lstset{style=python}

\AtBeginSection[]{
    \begin{frame}
        \vfill
        \centering
        \begin{beamercolorbox}[sep=8pt,center,shadow=true,rounded=true]{title}
            \usebeamerfont{title}\insertsectionhead\par%
        \end{beamercolorbox}
        \vfill
    \end{frame}
}


\setbeamercolor{palette primary}{bg=UIUCorange,fg=white}
\setbeamercolor{palette secondary}{bg=UIUCblue,fg=white}
\setbeamercolor{palette tertiary}{bg=UIUCblue,fg=white}
\setbeamercolor{palette quaternary}{bg=UIUCblue,fg=white}
\setbeamercolor{structure}{fg=UIUCorange} % itemize, enumerate, etc
\setbeamercolor{section in toc}{fg=UIUCblue} % TOC sections

\setbeamercolor{subsection in head/foot}{bg=UIUCorange,fg=UIUCblue}
\setbeamercolor{subsection in head/foot}{bg=UIUCorange,fg=UIUCblue}

\usepackage[utf8]{inputenc}
\usepackage{graphicx}


%Information to be included in the title page:
\title{\textbf{Topic 5: Conditionals}}
\author{\textbf{David H Smith IV}}
\institute[\textbf{UIUC}]{\textbf{University of Illinois Urbana-Champaign}}
\date{\textbf{Mon, July 05 2021}}

\setbeamertemplate{title page}[default][colsep=-4bp,rounded=true]
\addtobeamertemplate{title page}{\vspace{3\baselineskip}}{}
\addtobeamertemplate{title page}{
  \begin{textblock*}{\paperwidth}(-1.0em, -1.2em)
    \includegraphics[width=\paperwidth, height=\paperheight]{imgs/uiuc.png}
  \end{textblock*} 
}{}

\begin{document}

\frame{\titlepage}

\section{Announcements}

%
% Slide 1
%
\begin{frame}
  \frametitle{Weekly Reminders}
  \begin{enumerate}[A]
    \item Don't forget to register with the CBTF to take the quiz this week.
    \item Homework 4 is posted and homework 3 has entered it's grace period week.
    \item Post reading 6 due tomorrow.
  \end{enumerate}
\end{frame}


\section{Poll Questions}

%
% Slide 2
%
\begin{frame}[fragile]
  \frametitle{Poll Question: Boolean Expressions}
  Expressions that evaluate to \lstinline|True| or \lstinline|False|. 
  \begin{lstlisting}[language=Python, autogobble]
  (1 + 6) < (2 + 5)
  \end{lstlisting}
  \vfill
  \begin{enumerate}[A]
    \item \lstinline|True|
    \item \lstinline|False|
    \item TypeError
    \item SyntaxError
  \end{enumerate}
\end{frame}

%
% Slide 3
%
\begin{frame}[fragile]
  \frametitle{Poll Question: Boolean Expressions}
  Expressions that evaluate to \lstinline|True| or \lstinline|False|. 
  \begin{lstlisting}[language=Python, autogobble]
  "cat" < "Dog"
  \end{lstlisting}
  \vfill
  \begin{enumerate}[A]
    \item \lstinline|True|
    \item \lstinline|False|
    \item TypeError
    \item SyntaxError
  \end{enumerate}
\end{frame}

%
% Slide 2
%
\begin{frame}[fragile]
  \frametitle{Poll Question: If Statements}
  What does this code print?
  \begin{lstlisting}[language=Python, autogobble]
  x = 1
  if x < 7:
    print(x) 
  print(7)
  \end{lstlisting}
  \vfill
  \begin{enumerate}[A]
    \item 1
    \item 7
    \item \fbox{\parbox{0.02\textwidth}{1\\7}}
    \item SyntaxError
  \end{enumerate}
\end{frame}

%
% Slide 2
%
\begin{frame}[fragile]
  \frametitle{Poll Question: }
  What does this code print?
  \begin{lstlisting}[language=Python, autogobble]
  age = 17
  young = age < 30
  if young == true:
    print(age)
  \end{lstlisting}
  \vfill
  \begin{enumerate}[A]
    \item Nothing
    \item 17
    \item 30
    \item SyntaxError
  \end{enumerate}
\end{frame}

%
% Slide 2
%
\begin{frame}[fragile]
  \frametitle{Poll Question: If-Else Statements}
  What does this code print?
  \begin{lstlisting}[language=Python, autogobble]
  x = 2
  if x > 8:
    x = x - 2
    print(x)
  else:
    print(8)
  \end{lstlisting}
  \vfill
  \begin{enumerate}[A]
    \item 0
    \item 8
    \item \fbox{\parbox{0.02\textwidth}{0\\8}}
    \item SyntaxError
  \end{enumerate}
\end{frame}

%
% Slide 2
%
\begin{frame}[fragile]
  \frametitle{Poll Question: }

  \begin{lstlisting}[language=Python, autogobble]
  grade = 98
  if grade >= 90:
    print("You got an A!")
  if grade >= 80:
    print("You got a B!")
  else:
    print("You got something else")
  \end{lstlisting}
  \vfill
  \begin{enumerate}[A]
    \item You got an A\@!
    \item You got a B\@!
    \item You got something else
    \item The correct answer is not listed
  \end{enumerate}
\end{frame}

%
% Slide 2
%
\begin{frame}[fragile]
  \frametitle{Poll Question: }

  \begin{lstlisting}[language=Python, autogobble]
  grade = 98
  if grade >= 90:
    print("You got an A!")
  if grade >= 80 and grade < 90:
    print("You got a B!")
  else:
    print("You got something else")
  \end{lstlisting}
  \vfill
  \begin{enumerate}[A]
    \item You got an A\@!
    \item You got a B\@!
    \item You got something else
    \item The correct answer is not listed
  \end{enumerate}
\end{frame}

%
% Slide 2
%
\begin{frame}[fragile]
  \frametitle{Poll Question: If Statements}
  What's the result of running the following code?
  \begin{lstlisting}[language=Python, autogobble]
  x = 5
  if x == 3 or 4:
    print(x)
  \end{lstlisting}
  \vfill
  \begin{enumerate}[A]
    \item 3
    \item 4
    \item 5
    \item SyntaxError
  \end{enumerate}
\end{frame}

%
% Slide 2
%
\section{Boolean Operators}
\begin{frame}[fragile]
  \frametitle{Boolean Operators}
  \begin{enumerate}[A]
    \item Why is \lstinline|x == 3 or 4| always True?
    \item Alternatives:
      \begin{enumerate}
        \item \lstinline|x == 3 or x == 4|
        \item \lstinline|x in [3, 4]|
      \end{enumerate}
    \item Types of operators:
      \begin{enumerate}
        \item \textbf{Binary operators:} and, or
        \item \textbf{Unary Operators: } not
      \end{enumerate}
  \end{enumerate}
\end{frame}

%
% Slide 2
%
\begin{frame}[fragile]
  \frametitle{Truthy and Falsy}
  \begin{minipage}{0.69\textwidth}
    \begin{enumerate}[A]
      \item Python will convert non-Boolean types to Booleans. \\\lstinline|if "hello":|
      \item Accomplished via the use of the \lstinline|bool()| function. \\\lstinline|boo("hello")|
      \item All values are truthy (convert to \lstinline|True|) except those displayed to the right:
    \end{enumerate}
  \end{minipage}
  \begin{minipage}{0.29\textwidth}
    {\scriptsize
    \begin{itemize} 
      \item \lstinline|None|
      \item \lstinline|False|
      \item \lstinline|0|
      \item \lstinline|0.0|
      \item \lstinline|0j|
      \item \lstinline|Decimal(0)|
      \item \lstinline|Fraction(0, 1)|
      \item \lstinline|[]|
      \item \lstinline|{}|
      \item \lstinline|()|
      \item \lstinline|''|
      \item \lstinline|b''|
      \item \lstinline|set()|
      \item \lstinline|range(0)|
  \end{itemize}}
  \end{minipage}
\end{frame}

%
% Slide 2
%
\section{More Poll Questions}
\begin{frame}[fragile]
  \frametitle{Poll Question: }
  What does \lstinline|test(7)| return?
  \begin{lstlisting}[language=Python, autogobble]
  def test(num):
    if num > 0:
      return True
    return False
  \end{lstlisting}
  \vfill
  \begin{enumerate}[A]
    \item \lstinline|True|
    \item \lstinline|False|
    \item First \lstinline|True| then \lstinline|False|.
    \item The tuple \lstinline|(True, False)|
  \end{enumerate}
\end{frame}

%
% Slide 2
%
\begin{frame}[fragile]
  \frametitle{Poll Question: Printing with Bools}
  What does the following segment of code produce?
  \begin{lstlisting}[language=Python, autogobble]
  print("George") and print("Boole")
  \end{lstlisting}
  \vfill
  \begin{enumerate}[A]
    \item \fbox{\parbox{0.1\textwidth}{George}}
    \item \fbox{\parbox{0.1\textwidth}{Boole}}
    \item \fbox{\parbox{0.1\textwidth}{George\\Boole}}
    \item SyntaxError
  \end{enumerate}
\end{frame}

%
% Slide 2
%
\section{Short Circuit}
\begin{frame}[fragile]
  \frametitle{Short Circuiting}
  \begin{itemize}
    \item Python is \textbf{lazy} 
    \item It won't evaluate Boolean expressions it doesn't need to
  \end{itemize}
  \vfill
  \begin{lstlisting}[language=Python, autogobble]
  True or anything() # This is True 
  False and anything() # This is False
  \end{lstlisting}
  \vfill
  \begin{itemize}
    \item Python won't evaluate the \lstinline|anything()| part.
    \item You can use this to prevent errors from occurring in your code or having to next if statements:
  \end{itemize}
  \vfill
  \begin{lstlisting}[language=Python, autogobble]
  if (len(my_str) > 10) and (my_str[10] == 'a'):
    print("the tenth character of my string is ", my_str[10])
  \end{lstlisting}
  \vfill
\end{frame}

%
% Slide 2
%
\begin{frame}[fragile]
  \frametitle{Poll Question: }

  \begin{lstlisting}[language=Python, autogobble]

  \end{lstlisting}
  \vfill
  \begin{enumerate}[A]
    \item 
  \end{enumerate}
\end{frame}

%
% Slide 2
%
\begin{frame}[fragile]
  \frametitle{Poll Question: }

  \begin{lstlisting}[language=Python, autogobble]

  \end{lstlisting}
  \vfill
  \begin{enumerate}[A]
    \item 
  \end{enumerate}
\end{frame}

%
% Slide 2
%
\begin{frame}[fragile]
  \frametitle{Poll Question: }

  \begin{lstlisting}[language=Python, autogobble]

  \end{lstlisting}
  \vfill
  \begin{enumerate}[A]
    \item 
  \end{enumerate}
\end{frame}



\end{document}
