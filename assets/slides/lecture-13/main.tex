\documentclass{beamer}

\usepackage{graphicx}
\usepackage{textpos}
\usepackage{listings}
\usepackage{lstautogobble}

\usetheme{Madrid}
\useoutertheme{miniframes} % Alternatively: miniframes, infolines, split

% Setup the university's color pallette
\definecolor{UIUCorange}{RGB}{19, 41, 75} % UBC Blue (primary)
\definecolor{UIUCblue}{RGB}{232, 74, 39} % UBC Grey (secondary)

\definecolor{codegreen}{rgb}{0,0.6,0}
\definecolor{codegray}{rgb}{0.5,0.5,0.5}
\definecolor{codepurple}{rgb}{0.58,0,0.82}
\definecolor{backcolour}{rgb}{0.95,0.95,0.92}

\lstdefinestyle{python}{
  backgroundcolor=\color{backcolour},   
  commentstyle=\color{codegreen},
  keywordstyle=\color{magenta},
  numberstyle=\tiny\color{codegray},
  stringstyle=\color{codepurple},
  basicstyle=\ttfamily\footnotesize,
  breakatwhitespace=false,         
  belowskip=-0.5em,
  breaklines=true,                 
  captionpos=b,                    
  keepspaces=true,                 
  %numbers=left,                    
  numbersep=5pt,                  
  showspaces=false,                
  showstringspaces=false,
  showtabs=false,                  
  tabsize=2
}

\lstset{style=python}

\AtBeginSection[]{
    \begin{frame}
        \vfill
        \centering
        \begin{beamercolorbox}[sep=8pt,center,shadow=true,rounded=true]{title}
            \usebeamerfont{title}\insertsectionhead\par%
        \end{beamercolorbox}
        \vfill
    \end{frame}
}
% Setup the university's color pallette
\definecolor{UIUCorange}{RGB}{19, 41, 75} % UBC Blue (primary)
\definecolor{UIUCblue}{RGB}{232, 74, 39} % UBC Grey (secondary)


\setbeamercolor{palette primary}{bg=UIUCorange,fg=white}
\setbeamercolor{palette secondary}{bg=UIUCblue,fg=white}
\setbeamercolor{palette tertiary}{bg=UIUCblue,fg=white}
\setbeamercolor{palette quaternary}{bg=UIUCblue,fg=white}
\setbeamercolor{structure}{fg=UIUCorange} % itemize, enumerate, etc
\setbeamercolor{section in toc}{fg=UIUCblue} % TOC sections

\setbeamercolor{subsection in head/foot}{bg=UIUCorange,fg=UIUCblue}
\setbeamercolor{subsection in head/foot}{bg=UIUCorange,fg=UIUCblue}

\usepackage[utf8]{inputenc}
\usepackage{graphicx}

%Information to be included in the title page:
\title{\textbf{Dictionaries and Functions}}
\author{\textbf{David H Smith IV}}
\institute[\textbf{UIUC}]{\textbf{University of Illinois Urbana-Champaign}}
\date{\textbf{Mon, July 19 2021}}

\setbeamertemplate{title page}[default][colsep=-4bp,rounded=true]
\addtobeamertemplate{title page}{\vspace{3\baselineskip}}{}
\addtobeamertemplate{title page}{
  \begin{textblock*}{\paperwidth}(-1.0em, -1.2em)
    \includegraphics[width=\paperwidth, height=\paperheight]{imgs/uiuc.png}
  \end{textblock*} 
}{}

\begin{document}

\frame{\titlepage}

\section{Course Overview}

%
% Slide 1
%
\begin{frame}
  \frametitle{Reminders}
  \begin{itemize}
    \item Signup for the quiz or submit a conflict request
    \item Practice quiz is up
  \end{itemize}
\end{frame}

\section{Mini Review}

%
% Slide 2
%
\begin{frame}[fragile]
  \frametitle{Poll Question: Functions} 
  \vfill
  How many values will this function return if foo is called and bar is of type string?
  \begin{lstlisting}[language=Python, autogobble]
  def foo(bar):
    if type(foo) == str:
      return "lion"
    return "cheetah"
  \end{lstlisting}
  \vfill
  \begin{enumerate}[A]
    \item 0
    \item 1
    \item 2
    \item Another number
  \end{enumerate}
\end{frame}

%
% Slide 2
%
\begin{frame}[fragile]
  \frametitle{Poll Question: List Copying} 
  How many list objects exist after this code executes?
  \vfill
  \begin{lstlisting}[language=Python, autogobble]
  w = [1, 2, 3, 4]
  x = w
  y = x[:]
  z = y[0]
  \end{lstlisting}
  \vfill
  \begin{enumerate}[A]
    \item 0
    \item 1
    \item 2
    \item 3
  \end{enumerate}
\end{frame}

%
% Slide 2
%
\begin{frame}[fragile]
  \frametitle{Poll Question: Boolean} 
  What type does the following function return?
  \vfill
  \begin{lstlisting}[language=Python, autogobble]
  def qux(baz):
    return baz[0] == 'folded'
  \end{lstlisting}
  \vfill
  \begin{enumerate}[A]
    \item list
    \item integer
    \item boolean
    \item string
  \end{enumerate}
\end{frame}

%
% Slide 2
%
\begin{frame}[fragile]
  \frametitle{Poll Question: Sorting} 
  What is the value of \lstinline|z| after this code has executed?
  \vfill
  \begin{lstlisting}[language=Python, autogobble]
  x = [9, 12, 13, 4, 8, 13, 19, 6]
  z = x.sort()[0]
  \end{lstlisting}
  \vfill
  \begin{enumerate}[A]
    \item 0
    \item 4
    \item 19
    \item Error 
  \end{enumerate}
\end{frame}

%
% Slide 2
%
\begin{frame}[fragile]
  \frametitle{Sorting} 
  \begin{itemize}
    \item \lstinline|list.sort()| \textrightarrow\ Updates the \textit{original} list in place and returns \lstinline|None|.
    \item \lstinline|sorted()| Doesn't modify the original list. Returns a \textit{copy} of the list that is sorted.
  \end{itemize}
\end{frame}

\section{Common Errors in EiPE Questions}

%
% Slide 2
%
\begin{frame}[fragile]
  \frametitle{Code Reading Error Types} 
  \begin{minipage}{0.265\textwidth}
    \textbf{Error 1}
    \begin{lstlisting}[language=Python, autogobble]
      def f(x, y):
        z = 0
        for val in x:
          if val==y:
            z += 1
        return z
    \end{lstlisting}
    \vfill
  \end{minipage}
  \begin{minipage}{0.3\textwidth}
    \textbf{Error 2}
    \begin{lstlisting}[language=Python, autogobble]
    # x is a string
    # y/z are single character strings
    def f(x, y, z):
      k = x.find(y)
      y = x.find(z)
      return x[k+1: j]
    \end{lstlisting}
    \vfill
  \end{minipage}
  \begin{minipage}{0.415\textwidth}
    \textbf{Error 3}
    \begin{lstlisting}[language=Python, autogobble]
    def f(x):
      for i in range(len(x)):
        if x[i] % 2 == 0:
          x[i] = 2 * x[i]
    \end{lstlisting}
    \vfill
  \end{minipage}
  \vfill
  \begin{enumerate}[A]
    \item \textbf{Error 1 - Too Low Level: } for every value in x, add 1 to z if that value is equal to y,, then return z.
    \item \textbf{Error 2 - Type Mismatch: } returns the slice of a string from index y, incremented by 1, until, but not including index z.
    \item \textbf{Error 3 - Ambiguity/Imprecision: } Double the even indexes in list x
  \end{enumerate}
\end{frame}


\section{Dictionaries}

%
% Slide 2
%
\begin{frame}[fragile]
  \frametitle{In Review} 
  \vfill
  \begin{minipage}{0.60\textwidth}
    \begin{itemize}
      \item Consists of \lstinline|key:value| pairs.
      \item Why do we care? Tracking relationships between things. 
    \end{itemize}
  \end{minipage}
  \begin{minipage}{0.39\textwidth} 
    \begin{lstlisting}[language=Python, autogobble]
  name_map = {
    "dhsmith2" : {
      "first" : "David",
      "second" : "Smith"
    },
    "zilles" : {
      "first" : "Craig",
      "second" : "Zilles"
    }
  }\end{lstlisting}
  \end{minipage}
  \vfill
  \begin{lstlisting}[language=Python, autogobble]
    def informal_email(names_dict, netid):
      email = "Dear {0}, I wanted you to know ..."
      return email.format(names_dict[netid]['first'])

    email_text = informal_email(name_map, "dhsmith2")
  \end{lstlisting}
\end{frame}

%
% Slide 2
%
\begin{frame}[fragile]
  \frametitle{Poll Question: Dictionaries}
  What is the value of \lstinline|x| after the following function is called?
  \begin{lstlisting}[language=Python, autogobble]
  def get_item_counts(some_list):
    counts = {}
    for item in some_list:
        counts[item] += 1

  x = get_item_counts(["This", "This", "This", "Is", "A"])
  \end{lstlisting}
  \vfill
  \begin{enumerate}
    \item \lstinline|{"This": 3, "Is": 1, "A": 1}|
    \item \lstinline|{This: 3, Is: 1, A: 1}|
    \item \lstinline|{"This": "3", "Is": "1", "A": "1"}|
    \item KeyError
  \end{enumerate}
\end{frame}

%
% Slide 2
%
\begin{frame}[fragile]
  \frametitle{Dictionaries: Computing a Histogram}
  Creating a count map of items in a collection is a common dictionary pattern:
  \begin{lstlisting}[language=Python, autogobble]
  def get_item_counts(some_list):
    counts = {}
    for item in some_list:
      if item not in counts:
        counts[item] = 1
      else:
        counts[item] += 1
  \end{lstlisting}
\end{frame}

%
% Slide 2
%
\begin{frame}[fragile]
  \frametitle{Dictionary Functions}
  Functions for iteration:
  \begin{enumerate}
    \item \lstinline|dict.items()| \textrightarrow \ Generates tuples of all of the key value pairs in the dictionary.
    \item \lstinline|dict.keys()| \textrightarrow \ Generates all of the keys in the dictionary.
    \item \lstinline|dict.values()| \textrightarrow \ Generates all of the values in the dictionary.
  \end{enumerate}
  \vfill
  Functions for modification:
  \begin{enumerate}
    \item \lstinline|dict.clear()| \textrightarrow \ Clears all the key value pairs from the dictionary.
    \item \lstinline|dict.get(key, default)| \textrightarrow \ Tries to lookup the value associated with a key and gives default if key not found.
    \item \lstinline|dict1.update(dict2)| \textrightarrow \ Merges the key:value pairs from dict1 into dict2. 
    \item \lstinline|dict.pop(key, default)| \textrightarrow \ Removes the key:value pair, returns the value, default if key not found.
  \end{enumerate}
\end{frame}


\section{Functions}
%
% Slide 2
%
\begin{frame}[fragile]
  \frametitle{Poll Question: Function Parameters}
  What is the value of \lstinline|result| after the following code is run?
  \begin{lstlisting}[language=Python, autogobble]
    def arithmetic(x = 2, y = 3, z = 4):
      return x * y * z
    result = arithmetic(1, z=5)
  \end{lstlisting}
  \vfill
  \begin{enumerate}[A]
    \item 10
    \item 20
    \item 24
    \item 30
    \item Some other value
    \item Error
  \end{enumerate}
\end{frame}

%
% Slide 2
%
\begin{frame}[fragile]
  \frametitle{Function Returns}
  Functions can (kind of) return multiple values by putting them all in a tuple.
  \begin{lstlisting}[language=Python, autogobble]
  def return_first_and_lst(a_list):
    return (a_list[0], a_list[-1])
  x, y = return_first_and_last([1, 2, 3])
  \end{lstlisting}
\end{frame}

%
% Slide 2
%
\begin{frame}[fragile]
  \frametitle{Poll Question: Function Args and Mutability}
  What is the value of \lstinline|x[0]| after this code executes:
  \begin{lstlisting}[language=Python, autogobble]
  def remove_first(a_list):
    a_list = a_list[1:]

  x = [1, 2, 3, 4]
  remove_first(x)
  \end{lstlisting}
  \vfill
  \begin{enumerate}[A]
    \item 1
    \item 2
    \item 3
    \item 4
    \item Something else
    \item An error occurs
  \end{enumerate}
\end{frame}

%
% Slide 2
%
\begin{frame}[fragile]
  \frametitle{Poll Question: Functions}
  What, if anything, gets printed to the screen after this code executes?
  \begin{lstlisting}[language=Python, autogobble]
  def add1(x): return x + 1
  def mul2(x): return x * 2

  x = 1
  fns = [add1, mul2, mul2, print]
  for f in fns:
    x = f(x)
  \end{lstlisting}
  \vfill
  \begin{enumerate}[A]
    \item Nothing 
    \item 1
    \item 4
    \item 8
    \item Something else
    \item SyntaxError
  \end{enumerate}
\end{frame}

%
% Slide 2
%
\begin{frame}[fragile]
  \frametitle{Poll Question: Scope}
  What are the values of \lstinline|w, x, y| and \lstinline|z| after this code executes?
  \begin{lstlisting}[language=Python, autogobble]
  x, y, z = (7, 5, 10)

  def a_function(y):
    x = 2 * y
    return x * z

  w = a_function(x)
  \end{lstlisting}
  \vfill
  \begin{enumerate}[A]
    \item \lstinline|50, 7, 5, 10|
    \item \lstinline|100, 10, 5, 10|
    \item \lstinline|140, 7, 5, 10|
    \item \lstinline|140, 14, 5, 10|
    \item SyntaxError
  \end{enumerate}
\end{frame}

\end{document}
