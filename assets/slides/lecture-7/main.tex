\documentclass{beamer}

\usepackage{graphicx}
\usepackage{textpos}
\usepackage{listings}
\usepackage{lstautogobble}

\usetheme{Madrid}
\useoutertheme{miniframes} % Alternatively: miniframes, infolines, split

% Setup the university's color pallette
\definecolor{UIUCorange}{RGB}{19, 41, 75} % UBC Blue (primary)
\definecolor{UIUCblue}{RGB}{232, 74, 39} % UBC Grey (secondary)

\definecolor{codegreen}{rgb}{0,0.6,0}
\definecolor{codegray}{rgb}{0.5,0.5,0.5}
\definecolor{codepurple}{rgb}{0.58,0,0.82}
\definecolor{backcolour}{rgb}{0.95,0.95,0.92}

\lstdefinestyle{python}{
  backgroundcolor=\color{backcolour},   
  commentstyle=\color{codegreen},
  keywordstyle=\color{magenta},
  numberstyle=\tiny\color{codegray},
  stringstyle=\color{codepurple},
  basicstyle=\ttfamily\footnotesize,
  breakatwhitespace=false,         
  belowskip=-0.5em,
  breaklines=true,                 
  captionpos=b,                    
  keepspaces=true,                 
  numbers=left,                    
  numbersep=5pt,                  
  showspaces=false,                
  showstringspaces=false,
  showtabs=false,                  
  tabsize=2
}

\lstset{style=python}

\AtBeginSection[]{
    \begin{frame}
        \vfill
        \centering
        \begin{beamercolorbox}[sep=8pt,center,shadow=true,rounded=true]{title}
            \usebeamerfont{title}\insertsectionhead\par%
        \end{beamercolorbox}
        \vfill
    \end{frame}
}
% Setup the university's color pallette
\definecolor{UIUCorange}{RGB}{19, 41, 75} % UBC Blue (primary)
\definecolor{UIUCblue}{RGB}{232, 74, 39} % UBC Grey (secondary)


\setbeamercolor{palette primary}{bg=UIUCorange,fg=white}
\setbeamercolor{palette secondary}{bg=UIUCblue,fg=white}
\setbeamercolor{palette tertiary}{bg=UIUCblue,fg=white}
\setbeamercolor{palette quaternary}{bg=UIUCblue,fg=white}
\setbeamercolor{structure}{fg=UIUCorange} % itemize, enumerate, etc
\setbeamercolor{section in toc}{fg=UIUCblue} % TOC sections

\setbeamercolor{subsection in head/foot}{bg=UIUCorange,fg=UIUCblue}
\setbeamercolor{subsection in head/foot}{bg=UIUCorange,fg=UIUCblue}

\usepackage[utf8]{inputenc}
\usepackage{graphicx}


%Information to be included in the title page:
\title{\textbf{Topic 6: Loops}}
\author{\textbf{David H Smith IV}}
\institute[\textbf{UIUC}]{\textbf{University of Illinois Urbana-Champaign}}
\date{\textbf{Wed, July 07 2021}}

\setbeamertemplate{title page}[default][colsep=-4bp,rounded=true]
\addtobeamertemplate{title page}{\vspace{3\baselineskip}}{}
\addtobeamertemplate{title page}{
  \begin{textblock*}{\paperwidth}(-1.0em, -1.2em)
    \includegraphics[width=\paperwidth, height=\paperheight]{imgs/uiuc.png}
  \end{textblock*} 
}{}

\begin{document}

\frame{\titlepage}

\section{Reminders \& Updates}

%
% Slide 1
%
\begin{frame}
  \frametitle{Course Outcomes}
  \begin{itemize}
    \item Don't forget to register for this week's quiz with the CBTF.
    \item Start homework 4 sooner than later! 
    \item Be sure to try the practice quiz at least a few times before the real quiz.
    \item Be sure to finish zyBooks 6
  \end{itemize}
\end{frame}

\section{For Loops}

%
% Slide 2
%
\begin{frame}[fragile]
  \frametitle{Poll Question: For Loops}
  How many lines are printed to the screen?
  \begin{lstlisting}[language=Python, autogobble]
  course_times = {'CS 105': 'F9-11', "CS 125": "MWF11-12"}
  for course in course_times:
    print(course, 'meets', course_times[course])
  \end{lstlisting}
  \vfill
  \begin{enumerate}[A]
    \item 0
    \item 2
    \item 4
    \item KeyError
  \end{enumerate}
\end{frame}

%
% Slide 2
%
\begin{frame}[fragile]
  \frametitle{Poll Question: For Loop}
  How many lines will be printed?
  \begin{lstlisting}[language=Python, autogobble]
  things = [22, [33, 44], 55, [66]]
  for thing in things:
    print(thing)
  \end{lstlisting}
  \vfill
  \begin{enumerate}[A]
    \item SyntaxError
    \item 5
    \item 4
    \item TypeError
  \end{enumerate}
\end{frame}

%
% Slide 2
%
\begin{frame}[fragile]
  \frametitle{Poll Question: For Loop and Range}
  How many lines are printed to the screen?
  \begin{lstlisting}[language=Python, autogobble]
  for i in range(0, 10):
    print(i)
  \end{lstlisting}
  \vfill
  \begin{enumerate}[A]
    \item 11
    \item 10
    \item 9
    \item TypeError
  \end{enumerate}
\end{frame}

%
% Slide 2
%
\begin{frame}[fragile]
  \frametitle{Poll Question: For Loop and Range}
  How many lines are printed to the screen?
  \begin{lstlisting}[language=Python, autogobble]
  for i in range(-3, 9, 4):
    print(i)
  \end{lstlisting}
  \vfill
  \begin{enumerate}[A]
    \item 3
    \item 4
    \item 5
    \item another number\dots
  \end{enumerate}
\end{frame}

%
% Slide 2
%
\begin{frame}[fragile]
  \frametitle{Poll Question: Nested For Loops}
  How many lines are printed to the screen?
  \begin{lstlisting}[language=Python, autogobble]
  list1 = ['lemon', 'orange', 'lime']
  list2 = ['banana', 'lemon']

  for thing1 in list1:
    for thing2 in list2:
      print(thing2)
  \end{lstlisting}
  \vfill
  \begin{enumerate}[A]
    \item 5
    \item 6
    \item 7
    \item SyntaxError
  \end{enumerate}
\end{frame}

%
% Slide 2
%
\begin{frame}[fragile]
  \frametitle{Poll Question: Unpacking}
  What is the value of z?
  \begin{lstlisting}[language=Python, autogobble]
  x, y, z = [22, [33, 44], [66]]
  \end{lstlisting}
  \vfill
  \begin{enumerate}[A]
    \item 22
    \item 33
    \item \lstinline|[33, 44]|
    \item \lstinline|[66]|
  \end{enumerate}
\end{frame}

%
% Slide 2
%
\begin{frame}[fragile]
  \frametitle{Poll Question: Enumerate}
  What is the contents of new list?
  \begin{lstlisting}[language=Python, autogobble]
  orig_list = [3, 7, 22, 90]
  new_list = []
  for index, value in enumerate(orig_list):
    if (index % 2) == 0:
      new_list.append(value)
  \end{lstlisting}
  \vfill
  \begin{enumerate}[A]
    \item \lstinline|[3, 7]|
    \item \lstinline|[3, 22]|
    \item \lstinline|[3, 7, 22, 90]|
    \item \lstinline|[7, 90]|
  \end{enumerate}
\end{frame}

\section{While Loops}

%
% Slide 2
%
\begin{frame}[fragile]
  \frametitle{Poll Question: While Loops}
  How many lines are printed?
  \begin{lstlisting}[language=Python, autogobble]
  num = 14
  while num >= 1:
    print(num)
    num = num // 2
  \end{lstlisting}
  \vfill
  \begin{enumerate}[A]
    \item 3
    \item 4
    \item 5
    \item 7
  \end{enumerate}
\end{frame}

\section{Break vs Continue}
%
% Slide 2
%
\begin{frame}[fragile]
  \frametitle{Poll Question: Break}
  How many chars are printed?
  \begin{lstlisting}[language=Python, autogobble]
  for c in "sleepy":
    if c == "e":
      break
    print(c)
  \end{lstlisting}
  \vfill
  \begin{enumerate}[A]
    \item 4
    \item 1
    \item 2
    \item 6
  \end{enumerate}
\end{frame}

%
% Slide 2
%
\begin{frame}[fragile]
  \frametitle{Poll Question: Break vs Return}
  On which inputs do these functions behave differently?
  \centering
  \begin{minipage}{0.45\textwidth}
    \begin{lstlisting}[language=Python, autogobble]
    def func(a_list):
      for item in a_list:
        if item == "":
          break
        print(item)
      print("done")
    \end{lstlisting}
  \end{minipage}
  \hfill
  \begin{minipage}{0.45\textwidth}
    \begin{lstlisting}[language=Python, autogobble]
    def func(a_list):
      for item in a_list:
        if item == "":
          return 
        print(item)
      print("done")
    \end{lstlisting}
  \end{minipage}
  \vfill
  \begin{enumerate}[A]
    \item \lstinline|["a", "b", "", "d"]|
    \item \lstinline|["a", "b", "c", ""]|
    \item both
    \item neither
  \end{enumerate}
\end{frame}

%
% Slide 2
%
\begin{frame}[fragile]
  \frametitle{Poll Question: Continue}
  How many items are printed?
  \begin{lstlisting}[language=Python, autogobble]
  mixed_list = ['hi', '3', math.pi, 'there', ['CS', 105]]
  for item in mixed_list:
    if type(item) != str:
      continue
    print(item)
  \end{lstlisting}
  \vfill
  \begin{enumerate}[A]
    \item 0
    \item 3
    \item 2
    \item 6
  \end{enumerate}
\end{frame}

%
% Slide
%
\begin{frame}[fragile]
  \frametitle{Break vs Return vs Continue}
  \begin{itemize}
    \item \textbf{continue} \textrightarrow Skips everything below it and goes back to the beginning of the loop.
    \item \textbf{return} \textrightarrow Leaves function with return value.
    \item \textbf{break} \textrightarrow Exits loop it is apart of.
    \end{itemize}
\end{frame}


\end{document}
