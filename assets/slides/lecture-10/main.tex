\documentclass{beamer}

\usepackage{graphicx}
\usepackage{textpos}
\usepackage{listings}
\usepackage{lstautogobble}

\usetheme{Madrid}
\useoutertheme{miniframes} % Alternatively: miniframes, infolines, split

% Setup the university's color pallette
\definecolor{UIUCorange}{RGB}{19, 41, 75} % UBC Blue (primary)
\definecolor{UIUCblue}{RGB}{232, 74, 39} % UBC Grey (secondary)

\definecolor{codegreen}{rgb}{0,0.6,0}
\definecolor{codegray}{rgb}{0.5,0.5,0.5}
\definecolor{codepurple}{rgb}{0.58,0,0.82}
\definecolor{backcolour}{rgb}{0.95,0.95,0.92}

\lstdefinestyle{python}{
  backgroundcolor=\color{backcolour},   
  commentstyle=\color{codegreen},
  keywordstyle=\color{magenta},
  numberstyle=\tiny\color{codegray},
  stringstyle=\color{codepurple},
  basicstyle=\ttfamily\footnotesize,
  breakatwhitespace=false,         
  belowskip=-0.5em,
  breaklines=true,                 
  captionpos=b,                    
  keepspaces=true,                 
  numbers=left,                    
  numbersep=5pt,                  
  showspaces=false,                
  showstringspaces=false,
  showtabs=false,                  
  tabsize=2
}

\lstset{style=python}

\AtBeginSection[]{
    \begin{frame}
        \vfill
        \centering
        \begin{beamercolorbox}[sep=8pt,center,shadow=true,rounded=true]{title}
            \usebeamerfont{title}\insertsectionhead\par%
        \end{beamercolorbox}
        \vfill
    \end{frame}
}
% Setup the university's color pallette
\definecolor{UIUCorange}{RGB}{19, 41, 75} % UBC Blue (primary)
\definecolor{UIUCblue}{RGB}{232, 74, 39} % UBC Grey (secondary)


\setbeamercolor{palette primary}{bg=UIUCorange,fg=white}
\setbeamercolor{palette secondary}{bg=UIUCblue,fg=white}
\setbeamercolor{palette tertiary}{bg=UIUCblue,fg=white}
\setbeamercolor{palette quaternary}{bg=UIUCblue,fg=white}
\setbeamercolor{structure}{fg=UIUCorange} % itemize, enumerate, etc
\setbeamercolor{section in toc}{fg=UIUCblue} % TOC sections

\setbeamercolor{subsection in head/foot}{bg=UIUCorange,fg=UIUCblue}
\setbeamercolor{subsection in head/foot}{bg=UIUCorange,fg=UIUCblue}

\usepackage[utf8]{inputenc}
\usepackage{graphicx}

%Information to be included in the title page:
\title{\textbf{The Internet and Programming Patterns}}
\author{\textbf{David H Smith IV}}
\institute[\textbf{UIUC}]{\textbf{University of Illinois Urbana-Champaign}}
\date{\textbf{Mon, July 19 2021}}

\setbeamertemplate{title page}[default][colsep=-4bp,rounded=true]
\addtobeamertemplate{title page}{\vspace{3\baselineskip}}{}
\addtobeamertemplate{title page}{
  \begin{textblock*}{\paperwidth}(-1.0em, -1.2em)
    \includegraphics[width=\paperwidth, height=\paperheight]{imgs/uiuc.png}
  \end{textblock*} 
}{}

\begin{document}

\frame{\titlepage}

\section{Course Overview}

%
% Slide 1
%
\begin{frame}
  \frametitle{Reminders}
  \begin{itemize}
    \item Signup for the quiz or submit a conflict request
    \item Homework 6 is up 
    \item Practice quiz is up
    \item zyBooks and post reading are due tomorrow
  \end{itemize}
\end{frame}

\section{The Internet}

%
% Slide 2
%
\begin{frame}[fragile]
  \frametitle{Poll Question: Acronyms}
  When you request a web page your request is part of what protocol?
  \vfill
  \begin{enumerate}[A]
    \item HTML
    \item DNS
    \item CSS
    \item HTTP
    \item JS
    \item WWW
  \end{enumerate}
\end{frame}

%
% Slide 2
%
\begin{frame}[fragile]
  \frametitle{Poll Question: Acronyms}
  When you request a web page your request is part of what protocol?
  \vfill
  \begin{enumerate}[A]
    \item \textbf{HTML:} Hyper Text Markup Language
    \item \textbf{DNS:} Domain Name Service
    \item \textbf{CSS:} Cascading Style Sheets
    \item \textbf{HTTP:} Hyper Text Transfer Protocol
    \item \textbf{JS:} Javascript
    \item \textbf{WWW:} World Wide Web
  \end{enumerate}
\end{frame}

%
% Slide 2
%
\begin{frame}[fragile]
  \frametitle{Poll Question: The Internet}
  Which of the following are true?
  \vfill
  \begin{enumerate}[A]
    \item Tags in HTML documents typically come in pairs
    \item HTML documents are hierarchical
    \item CSS properties consists of attribute: value pairs
    \item The preferred method of styling web pages is through CSS
    \item Javascript can be used to modify both the content of a web page and its presentation
    \item All of the statements are true
  \end{enumerate}
\end{frame}

\section{Patterns}

%
% Slide 2
%
\begin{frame}[fragile]
  \frametitle{Counting Pattern}
  \begin{lstlisting}[language=Python, autogobble][language=Python, autogobble]
  def count(collection):
    counter = 0
    for item in collection:
      if <item meets condition>:
        counter += 1
    return counter
  \end{lstlisting}
\end{frame}

%
% Slide 2
%
\begin{frame}[fragile]
  \frametitle{Computing a Sum/Total}
  \begin{lstlisting}[language=Python, autogobble][language=Python, autogobble]
  def sum(collection):
    total = 0
    for item in collection:
      total += item
    return total
  \end{lstlisting}
\end{frame}

%
% Slide 2
%
\begin{frame}[fragile]
  \frametitle{Finding (single thing) in a Collection}
  \begin{lstlisting}[language=Python, autogobble][language=Python, autogobble]
  def find_thing(collection):
    for thing in collection:
      if <thing meets condition>:
        return thing
  \end{lstlisting}
  \vfill
  \begin{lstlisting}[language=Python, autogobble][language=Python, autogobble]
  def find_thing(collection):
    found = None
    for thing in collection:
      if <thing meets condition>:
        found = thing
        break
    return found
  \end{lstlisting}
\end{frame}

%
% Slide 2
%
\begin{frame}[fragile]
  \frametitle{Using Loop else when nothing found}
  \begin{lstlisting}[language=Python, autogobble][language=Python, autogobble]
  def find_thing(collection):
    for thing in collection:
      if <thing meets condition>:
        found = thing
        break
    else:
      found = <something>

   return found
  \end{lstlisting}
\end{frame}

%
% Slide 2
%
\begin{frame}[fragile]
  \frametitle{Finding best in collection}
  \begin{lstlisting}[language=Python, autogobble][language=Python, autogobble]
  def find_best(collection):
    currentbest = ??
    for thing in collection:
      if <thing is better than current best>:
        currentbest = thing
    return currentbest 
  \end{lstlisting}
  \vfill
  \begin{itemize}
    \item If we're searching over a list and we want to return the largest or smaller number: \lstinline|currentbest = stufflist[0]| \pause
    \item If we're searching over a list of strings and we want to return the longest string: \lstinline|currentbest = stufflist[0]| or \lstinline|currentbest = ""| \pause
    \item If you know the list contains only non-negative integers: \lstinline|currentbest = -1|
  \end{itemize}
\end{frame}

%
% Slide 2
%
\begin{frame}[fragile]
  \frametitle{Filtering a collection}
  \begin{lstlisting}[language=Python, autogobble][language=Python, autogobble]
  def filter(collection):
    new_list = []

    for thing in collection:
      if <thing meets criteria>:
        newlist.append(thing)

    return new_list
  \end{lstlisting}
\end{frame}

%
% Slide 2
%
\begin{frame}[fragile]
  \frametitle{Poll Question: Patterns}
  Given a list of names, make a new list containing only those enrolled in a given course.
  \begin{lstlisting}[language=Python, autogobble]
  def find_student_enrolled_in_course(student_registrations, course):
    students = []
    for student in student_registrations:
      if course in student_registrations[student]:
        students.append(student)
    return students
  \end{lstlisting}
  \begin{enumerate}[A]
    \item Sum
    \item Counter
    \item Finding ``best'' in collection
    \item Filtering a collection
    \item None of the above
  \end{enumerate}
\end{frame}

%
% Slide 2
%
\begin{frame}[fragile]
  \frametitle{Poll Question: Patterns}
  Given a list of strings find the longest string.
  \begin{lstlisting}[language=Python, autogobble]
  def find_longest_string(strings):
    longest_string = strings[0]
    for string in strings[1:]:
      if len(string) > len(longest_string):
        longest_string = string
    return longest_string
  \end{lstlisting}
  \vfill
  \begin{enumerate}[A]
    \item Sum
    \item Counter
    \item Finding ``best'' in collection
    \item Filtering a collection
    \item None of the above
  \end{enumerate}
\end{frame}

%
% Slide 2
%
\begin{frame}[fragile]
  \frametitle{Poll Question: Patterns}
  Given a list of strings, find the number of strings that contain the substring \lstinline|"tion"|.
  \begin{enumerate}[A]
    \item Sum
    \item Counter
    \item Finding ``best'' in collection
    \item Filtering a collection
    \item None of the above
  \end{enumerate}
  \vfill
  \begin{lstlisting}[language=Python, autogobble]
    def count_substring(strings):
      counter = 0
      for string in strings:
        if "tion" in string:
          counter += 1
      return counter
  \end{lstlisting}
\end{frame}

\end{document}
